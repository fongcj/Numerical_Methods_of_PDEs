\documentclass[12pt]{article}
\usepackage{defs}
\usepackage{times}
\title{Rubric for E4301 Projects}
\date{\today}
\author{Marc Spiegelman}
\begin{document}
\maketitle{}

\subsection*{Project Rationale}
\label{sec:project-rationale}

The purpose of the 4301 project is to  work through the entire process of
modeling, computation, verification and evaluation that is required to
solve quantitative problems in numerical PDE's, by working on a
problem of \emph{your} choice (with guidance from me).  As a guide, this
document will lay out a possible template that includes all of the
components of the project I am looking for and some notion of what I
am expecting.  A major theme of this course is making intelligent
choices of problems, formulations, techniques etc.,  so for each part of the
project I want you to justify your choices (or if they turn out to
have issues, which they often will, to discuss what went wrong and
what you would choose next and why).  The project outline is
essentially this set of choices and their justification.  

\subsection*{Project Outline}
\label{sec:project-outline}

\begin{enumerate}
\item Choose your problem and justify its importance/interest etc
\item Choose a mathematical model (PDE preferably)
  \begin{enumerate}
  \item Discuss its formulation
  \item Justify and explain any approximations made to make it more tractable
  \end{enumerate}
\item Choose and justify your Numerical methods including
  \begin{enumerate}
  \item Choice of discretization: e.g. Finite Difference, Finite
    Volume, Finite Element, something fancier?
  \item Choice of Linear/Non-linear Solver
  \item Choice of software for modeling/visualization/analysis
  \end{enumerate}
\item Present results on your implementation, what worked, what didn't
  and why. If possible, results should include basic convergence and
  regression tests as well as any arguments to justify that your
  numerical model is doing what it's supposed to be doing.
\item Discuss pros/cons of choices (including whether you think the
  model is useful);  what you might do differently next
  time; what the next steps might be if you continued the project.
\end{enumerate}



\subsection*{Project Proposal}
\label{sec:project-proposal}

For your project proposal, what I am looking for (at this point) is one
to two paragraphs discussing at least Points 1 and 2 above (i.e. what
problem do you want to solve and how are you going to sort out a PDE
model of the problem).  If you have a sense of numerical methods you
would like to try, that would also be very useful.

Overall I am quite flexible in the scope and direction of these
projects and am happy to work with you to flesh them out.  They can be
parts of ongoing research,  something you just always wanted to try,
or a starter problem for new research.  Pitch me an idea and we'll
take it from there.

To hand in your proposal, just add it to your bitbucket repository
under a directory called Project.  Ideally I should see a
\texttt{apma4301proposal\_<userid>.tex} and it's corresponding pdf
file.  Word docs are acceptible, but now is a good time to start
learning \LaTeX{} (on top of everything else you need to know).
\end{document}


%%% Local Variables: 
%%% mode: latex
%%% TeX-master: t
%%% End: 
